%
% The Utility of Computational Models for Understanding Autism
%

An important goal and contribution of my work is the use of a relatively novel tool in ASD research, the methods of computational cognitive neuroscience.  These methods provide a way to formalize how differences in the underlying neural hardware give rise to the patterns of behavior found in people with autism.  Specifically, we modified previously published and validated computational models of human behavior in accordance with my hypothesis of dysfunctional PFC / DA interactions in an attempt to capture the performance of people with autism.  Utilizing this approach we captured autistic behavior in the diverse areas of executive dysfunction, overselectivity\footnote{The overselectivity model was the only model that was not previously published.}, tasks of implicit learning, homograph disambiguation, and category learning.  Indeed, the strength of the work presented here is not in any single explanation of autistic behavior, but in providing a single, plausible, precise neural mechanism that is capable of providing a level of inter-theorhetic reduction previously not seen in ASD research.  However, at this point the concept of dysfunctional interactions between the mid-brain DA system and the PFC in people with autism is still just a theory, there is much work that is left to be done to either justify or modify my theory going forward. 

