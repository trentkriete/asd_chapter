%
% The Utility of Computational Models for Understanding Autism
%

\subsection{The Utility of Computational Cognitive Neuroscience}
An important contribution of this work involves the fact that it uses a relatively novel tool in ASD research: the methods of computational cognitive neuroscience. These methods provide a way to formalize how differences in the underlying neural circuitry give rise to the patterns of behavior found in people with autism. Specifically, we modified previously published and validated computational models of human behavior in accordance with the hypothesis that autism involves dysfunctional PFC/DA interactions. These modified models were shown to capture the behavioral performance of people with autism. This approach allowed us to offer a unified biological explanation for autistic behavior in the diverse areas of executive dysfunction, implicit learning tasks, homograph disambiguation, and category learning. Indeed, the strength of the work presented here is not in any single model of autistic behavior on any single task, but in providing a unified, plausible, precise neural mechanism that is capable of providing a level of inter-theorhetic reduction previously not seen in ASD research. This work provides an example of the important role that computational cognitive neuroscience can play in developing an understanding of autism and other developmental disorders.

\subsection{Previous Computational Models of Autism}
The formal and explicit nature of computational cognitive modeling suggests a novel approach to autism research. In order for computational models to be useful in this endeavor, they must be constrained by both bottom-up (neurobiological mechanisms) and by top-down (observed behavior) considerations, providing a formal characterization of the relationship between these levels of analysis. While there have been some previous computational models of ASD, it is not at all clear that they have offered such an explicit and detailed connection between biology and behavior.

Some previous computational models have attempted to address specific behavioral aspects of autism, including poor generalization~\cite{CohenIL:1994:AutismLearning,GustafssonL:1997:AutismMaps}, Weak Central Coherence~\cite{OLoughlinC:2000:Coherence}, and overselectivity~\cite{McClellandJL:2000:Autism}. One ambitious computational framework, developed by Grossberg \& Seidman (2006)~\nocite{RefWorks:146}, offers an explanation of multiple aspects of autism, including poor generalization, as well as cognitive and emotional issues.

A shortcoming of many of the existing models of autism is their fairly abstract nature, making little contact with specific neurobiological properties or measures~\cite{CohenIL:1994:AutismLearning,McClellandJL:2000:Autism,OLoughlinC:2000:Coherence}. Those models of autism which have incorporated biology into their framework have, thus far, only matched qualitative patterns of behavior rather than attempting to account for any quantitative behavioral data~\cite{GustafssonL:1997:AutismMaps,RefWorks:146}. Models that are more tightly coupled with observed functional properties of neurobiological systems, while being constrained by quantitative behavioral data, such as the models presented in this chapter, may have a more profound impact on our understanding of autism.
