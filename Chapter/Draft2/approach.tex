%
% General Modeling Approach
% 

One reasonable modeling strategy would be to create one expansive model that encompasses all phenomena of interest, demonstrating both how normal and autistic-like behavior may arise in the same framework given specific dopamine-related parameter adjustments.  However, the range of relevant behaviors that are affected by autism is incredibly varied making this approach nearly untennable.   Rather than building one massive model from scratch, separate previously published models, each considered to be one of the strongest representatives in their respective domains, were manipulated in a manner which is consistent with our general hypothesis.  Supplementing the previously published models, one completely novel and unpublished model, encompassing an important aspect of behavioral data in autism, was developed as well.  This approach allows me to demonstrate ---using the best models currently available--- how inflexible attentional switching, caused by deficient DA / PFC interactions, can facilitate and possibly cause the behavior seen in people with autism.  It is important to note that no additional mechanisms were introduced into any of the previously published models.  Only their existing, previously justified, mechanisms for cognitive flexibility were manipulated in order to capture the behavior of people with autism.  Using this approach, we demonstrate how cognitive inflexibility, arising from improper DA modulation of PFC, can account for a wide variety of behavioral phenomena observed in autism, including: poor generalization in concept learning tasks (e.g. prototype formation), a lack of context sensitivity in language processing, stimulus overselectivity, executive dysfunction, and deficits in implicit learning. 

