%
% General Modeling Approach
% 

This work focuses on using the methods of computational cognitive neuroscience to demonstrate how changes in PFC/DA interactions, leading to difficulties in updating PFC representations, can explain a broad array of behavioral patterns observed in autism. One reasonable strategy would involve the fabrication of a single complex biological model that is capable of performing all of the laboratory behaviors of interest, showing that this model matches the behavior of typically developing people but produces the behavioral patterns observed in ASD when it is modified to include PFC/DA dysfunction. Unfortunately, the range of relevant behaviors is broad, making the production of a single model capable of all of the behaviors of interest untennable.

As an alternative, we have pursued a strategy that builds directly upon the rich existing modeling literature. Separate previously published models, each successful in capturing behavioral patterns in a particular domain, were modified in a manner that reflects our general hypothesis that PFC/DA interactions are disrupted in ASD so as to reduce flexibility in PFC updating. In each case, we have found that hindering PFC updating causes the modified model to produce patterns of performance seen in ASD. It is important to note that no additional mechanisms were introduced into any of the previously published models. Only their existing, previously justified, mechanisms for cognitive flexibility were manipulated in order to capture the behavior of people with autism. Using this approach, we can show how cognitive inflexibility, arising from improper DA modulation of PFC, can account for a wide variety of behavioral phenomena observed in autism, including executive dysfunction, deficits on implicit learning tasks, problems with word sense disambiguation, and reduced prototype formation.

