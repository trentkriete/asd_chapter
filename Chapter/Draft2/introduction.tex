% 
% Introduction
%
% Last Modified: Tue Dec 29 21:16:36 PST 2015
%

Autism is a complex developmental disorder diagnosed by the presence of a triad of symptoms: social impairments, communication impairments, and repetitive stereotyped behaviors~\nocite{RefWorks:98}(DSM-IV-TR, 2000). The severity, and sometimes even the presence, of these symptoms varies greatly across those afflicted with the disorder. Due, in part, to this variability, autism is generally seen as a spectrum of disorders, collectively known as \emph{autism spectrum disorders} (ASD). Steady progress has been made in early diagnosis, as well as in the design of interventions that mitigate problematic behaviors. However, no consensus has been reached concerning the neural basis of autism. This chapter presents a theoretical framework that explains many aspects of autistic behavior in terms of particular neurological differences.  Specifically, \emph{computational cognitive neuroscience} modeling methods are used to demonstrate how dysfunctional interactions between the midbrain dopamine (DA) system and the prefrontal cortex (PFC) could give rise to many of the behavioral patterns seen in ASD.

People with autism exhibit difficulties on a range of cognitive tasks. These tasks assess such capabilities as flexible adaptation, planning toward a goal, the generation of novel ideas, and determining the mental states of others~\cite{BennettoL:1996:AutismPlanningWCST,Ozonoff:1999:AutismStroopWCST,TurnerW:1999:AutismGenerativity,Baron-Cohen:1985:AutismTOM}. Abnormal gaits, problems initiating movements, abnormal sleep patterns, and an increased likelihood of developing a seizure disorder all accompany an autism diagnosis~\cite{RefWorks:99,RefWorks:100,RefWorks:101,RefWorks:102}. Juxtaposed against the impairments of ASD exists a collection of spared, and sometimes enhanced, abilities. For example, superior perceptual discrimination is regularly seen on the \emph{embedded figures task} (EFT)~\cite{WitkinHA:1971:EFT,RefWorks:103,RefWorks:104}. Also, improved (and sometimes even savant) abilities have been observed in domains as diverse as mathematics, map memorization, music, artistic abilities, and date calculations~\cite{RefWorks:105,RefWorks:106,RefWorks:107,RefWorks:37}.   

This diverse behavioral profile poses a daunting challenge to the development of a unified theory of ASD. Indeed, the most widely acknowledged theories are generally circumscribed to account for only specific phenomena. For instance, the ``Theory of Mind'' (TOM) hypothesis~\cite{Baron-Cohen:1985:AutismTOM} asserts that people with autism lack the ability to understand mental states in others, offering an explanation of observed social difficulties. It is unclear, however, how the TOM hypothesis might explain non-social patterns of deficits and spared abilities. In comparison, the ``Weak Central Coherence'' (WCC) hypothesis~\cite{RefWorks:37}, which posits a more ``piecemeal'' style of cognitive processing in ASD, rather than one that is more ``holistic'', explains the appearance of enhanced performance on tasks that require attention to detail, along with difficulties utilizing more global, contextual, and gestalt information, but it does not address the full range of observed phenomena. Similarly, problems with planning, the flexible adaptation of behavior, and the generation of novel ideas are the focus of ``Executive Dysfunction'' (ED) theory, which highlights changes in executive processing as a central feature of autism~\cite{HughesC:1994:AutismExecutiveDysfunction,HillEL:2004:AutismExecutiveDysfunction,Ozonoff:1991:AutismExecutiveDysfunction}.  

Combining multiple theories of this kind might cover the behavioral landscape of ASD, but it is not clear that this approach will foster our understanding of the neural basis of the condition. There is good reason to be skeptical that the different broad domains addressed by these theories arise from distinct brain systems. Also, any neural account will need to explain exactly how biological differences, over the course of development, give rise to the diverse range of observed behavioral patterns.

An alternative approach involves the fabrication and analysis of computational models of neural processes that can simulate the generation of behaviors comparable to those observed in the laboratory. The methods of computational cognitive neuroscience produce formal characterizations of the relationship between brain and behavior, entailing precise and testable hypotheses involving both neuroscientific and psychological measures. By offering explicit mechanistic accounts of the underlying neurobiology, while capturing actual behavioral patterns, computational cognitive neuroscience models provide a means of bridging the conceptual gap between cognitive psychology and cognitive neuroscience in ASD research.

One question previously explored using computational cognitive neuroscience techniques is how deliberate control over behavior (cognitive control) is instantiated within neural circuitry and how this control is adjusted as environmental contingencies change (cognitive flexibility). The prefrontal cortex (PFC) has been broadly implicated in both cognitive control and cognitive flexibility~\cite{Stuss:2000:WCSTLesion,Stuss:2001:StroopLesion}. Under some accounts, cognitive control involves the active maintenance of abstract rule-like representations in PFC~\cite{NoelleDC:2012:Rules}. These PFC representations provide a top-down task-appropriate processing bias to more posterior brain areas~\cite{CohenJD:1990:Stroop,RefWorks:154}. Biologically, the active maintenance of frontal control representations is supported by dense patterns of recurrent excitation in the PFC~\cite{PucakML:1996:Stripes}, as well as intrinsic maintenance currents~\cite{Goldman-RakicPS:1987:PFC_Maintenance}. Computational analyses have shown that cognitive control and cognitive flexibility are, in a sense, at odds. Cognitive control requires robust maintenance of a control representation, while cognitive flexibility requires the ability to quickly adapt these representations as task contingencies change. This processing conflict suggests the need for a mechanism to intelligently toggle the PFC between a maintenance mode and an updating mode. The fact that we learn to control our behavior in different ways depending on the current situation means that this toggling process must be learned. This need for learning has drawn the attention of researchers to the dopamine system.   

Dopamine (DA), a neurotransmitter with diffuse projections throughout the brain, plays a central role in contemporary models of PFC function. The mesolimbic DA system is seen as implementing a reinforcement learning algorithm, driving the learning of action sequences that lead to reward~\cite{MontaguePR:1996:Dopamine,BartoAG:1994:TDLearning}. In PFC models, the DA system learns to adjust the state of PFC pyramidal cells, determining when cognitive control should be maintained and when it should be flexibly modified in order to succeed at the current task~\cite{BraverTS:2000:Control,RougierNP:2005:XT}. A useful analogy is that of a ``gate'' in a fenced enclosure. When cognitive control must be strong, the gate is closed, keeping out distracting inputs that might compromise the current PFC control signals. When the current control state is no longer appropriate, the gate opens, allowing the old control state to escape and permitting a new control representation to enter the PFC via its inputs.  Recent computational models of PFC function suggest that intelligent ``gating'' of control representations in PFC can be learned, through experience, via the DA system~\cite{RougierNP:2005:XT,RougierNP:2002:TaskSwitching}. 

We propose that these computational accounts of PFC/DA interactions are highly relevant for understanding the neural basis of ASD. There is growing evidence for abnormal DA functioning in people with autism. Aberrant levels of DA have been discovered in studies measuring DA via PET~\cite{FernellE:1997:AutismPET}, as well as more indirect measures such as HVA metaboloites~\cite{MartineauJ:1992:AutismDopamine}. Clinical trials of drugs that modulate levels of DA in the brain have shown behavioral benefits, as well~\cite{PoseyDJ:2000:AutismDopamine,TsaiLY:1999:AutismDopamine}. DA system dysfunction is also associated with behaviors related to ASD symptoms, incuding increased prevalence of seizures, repetitive behaviors, and problems with skilled motor learning~\cite{RefWorks:1,RefWorks:3,RefWorks:5,RefWorks:2,RefWorks:109}.

In this chapter, we report computational cognitive neuroscience simulation results supporting the conjecture that deficits in PFC/DA interactions are responsible for many of the interesting behavioral patterns observed in ASD. The basic idea is that reduced efficacy of the DA-based PFC gating mechanism results in overly perseverative top-down control. This has three results. First, the inability to properly adapt PFC control representations produces inflexible behavior. This is exemplified by the executive dysfunction profile observed in autism. The second consequence is more subtle. We suggest that the flexible updating of PFC plays an important role in shaping associational areas of cortex, influencing synaptic plasticity in these areas so as to support the appropriate generalization of learned behaviors across contexts. The hypothesized lack of flexible updating of PFC in ASD results in the learning of cortical representations that are overly specific, hindering generalization. These learning deficits can be seen both in tests of stimulus overselectivity during conditioning and in measures of prototype extraction during category learning. Third, failure to appropriately update PFC can limit the use of temporally extended context information, explaining observed deficits in sequential implicit learning tasks and in the use of sentential context to disambiguate the meaning of words.

% In the following we use computational modeling methods to investigate and formalize what effect a DA deficit would have on the behavior of a developing individual, relating simulated behavioral results to actual data from studies of people with autism.  

% In this chapter we investigate the degree to which a single deficit, a dysfunctional dopaminergic system, can account for many of the patterns of behavior demonstrated by people with autism.  This investigation makes use of the methods of computational cognitive neuroscience, producing formal models that demonstrate how dopamine dysfunction can produce the behavioral patterns of interest.  By focusing on a single neurological mechanism with diverse behavioral effects, this approach provides a level of inter-theoretic reduction not found in many current theories seeking to explain autism.  The implications of the research expand beyond fundamental theorizing, potentially providing an improved understanding of the successes and failures of interventions utilized to reduce overselectivity and foster the generalization of learned behaviors in people with ASD.

We begin by reviewing the processes of DA-mediated PFC updating, as well as previous computational modeling efforts that have focused on ASD. We then present a series of computational cognitive neuroscience models that demonstrate how dysfunctional PFC/DA interactions can account for experimental data concerning ASD behavioral patterns in the diverse domains of executive function, stimulus overselectivity, implicit learning, lexical dismbiguation, and prototype formation.
