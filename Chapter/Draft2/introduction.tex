% 
% Introduction
%
% Last Modified: Tue Dec 29 21:16:36 PST 2015
%

Autism is a complex developmental disorder diagnosed by the presence of a triad of symptoms: social impairments, communication impairments, and repetitive stereotyped behaviors~\nocite{RefWorks:98}(DSM-IV-TR, 2000). The severity, and sometimes even the presence, of these symptoms varies greatly across those afflicted with the disorder. Due, in part, to this variability, autism is generally seen as a spectrum of disorders, collectively known as \emph{autism spectrum disorders} (ASD). Steady progress has been made in early diagnosis, as well as in the design of interventions that mitigate problematic behaviors. However, no consensus has been reached concerning the neural basis of autism. This chapter presents a theoretical framework that explains many aspects of autistic behavior in terms of particular neurological differences.  Specifically, \emph{computational cognitive neuroscience} modeling methods are used to demonstrate how dysfunctional interactions between the midbrain dopamine (DA) system and the prefrontal cortex (PFC) could give rise to many of the behavioral patterns seen in ASD.

People with autism exhibit difficulties on a range of cognitive tasks. These tasks assess such capabilities as flexible adaptation, planning toward a goal, the generation of novel ideas, and determining the mental states of others~\cite{BennettoL:1996:AutismPlanningWCST,Ozonoff:1999:AutismStroopWCST,TurnerW:1999:AutismGenerativity,Baron-Cohen:1985:AutismTOM}. Abnormal gaits, problems initiating movements, abnormal sleep patterns, and an increased likelihood of developing a seizure disorder all accompany an autism diagnosis~\cite{RefWorks:99,RefWorks:100,RefWorks:101,RefWorks:102}. Juxtaposed against the impairments of ASD exists a collection of spared, and sometimes enhanced, abilities. For example, superior perceptual discrimination is regularly seen on the \emph{embedded figures task} (EFT)~\cite{WitkinHA:1971:EFT,RefWorks:103,RefWorks:104}. Also, improved (and sometimes even savant) abilities have been observed in domains as diverse as mathematics, map memorization, music, artistic abilities, and date calculations~\cite{RefWorks:105,RefWorks:106,RefWorks:107,RefWorks:37}.   

This diverse behavioral profile poses a daunting challenge to the development of a unified theory of ASD. Indeed, the most widely acknowledged theories are generally circumscribed to account for only specific behavioral phenomena. For instance, the ``Theory of Mind'' (TOM) hypothesis~\cite{Baron-Cohen:1985:AutismTOM} asserts that people with autism lack the ability to understand mental states in others, offering an explanation of observed social difficulties. It is unclear, however, how the TOM hypothesis might explain non-social patterns of deficits and spared abilities. In comparison, the ``Weak Central Coherence'' (WCC) hypothesis~\cite{RefWorks:37}, which posits a more ``piecemeal'' style of cognitive processing in ASD, rather than one that is more ``holistic'', explains the appearance of enhanced performance on tasks that require attention to detail, along with difficulties utilizing more global, contextual, and gestalt information, but it does not address the full range of observed phenomena. Similarly, problems with planning, the flexible adaptation of behavior, and the generation of novel ideas are the focus of ``Executive Dysfunction'' (ED) theory, which highlights changes in executive processing as a central feature of autism~\cite{HughesC:1994:AutismExecutiveDysfunction,HillEL:2004:AutismExecutiveDysfunction,Ozonoff:1991:AutismExecutiveDysfunction}.  

Combining multiple theories of this kind might cover the behavioral landscape of ASD, but it is not clear that this approach will foster our understanding of the neural basis of the condition. There is good reason to be skeptical that the different broad domains addressed by these theories arise from distinct brain systems. Also, any neural account will need to explain exactly how biological differences, over the course of development, give rise to the diverse range of observed behavioral patterns.

Given these concerns, one might opt to pursue an explanation of ASD that begins with identified differences in brain structure and/or function, using the existent literature on functional localization in order to associate these differences with observed patterns of behavior. One of the primary obstacles to this approach is the fact that a diverse range of neurological differences have been discovered in ASD.

Abnormalities in the structure of the cerebellum are frequently found in ASD~\cite{AkshoomoffNA:2000:Neurological,RefWorks:133}. These findings consist of hypoplasia (reduced growth) and hyperplasia (increased growth)~\cite{RodierPM:1996:AutismCerebellum} within the cerebellar vermis. Dysfunction of the cerebellum may account for some of the motor difficulties found in people with autism, since the cerebellum is known to be critical for coordinated motor control.  Interestingly, the cerebellum may also be important for shifting attention~\cite{RefWorks:135,CourchesneE:1994:CerebellumAttentionShift}. Studies investigating this possibility have shown that people with autism and patients with cerebellar lesions perform similarly on tasks requiring the ability to rapidly shift attention between different stimulus dimensions~\cite{CourchesneE:1994:CerebellumAttentionShift,RefWorks:136,RefWorks:134}. Other brain systems, including circuits involving the prefrontal cortex (PFC) and the midbrain dopamine (DA) system, are also involved in the shifting of perceptual attention, however, leaving some uncertainty concerning the source of the deficits observed in ASD. For instance, a study comparing patients with cerebellar dysfunction to those with Parkinson's disease (which degrades DA function) found distinguishable deficits on tasks requiring shifts of attention~\cite{RefWorks:145}. While both patient populations performed worse than control groups, only the group with cerebellar pathology showed improvement when the motor demands of the task were reduced. This suggests that observed attentional deficits arising from cerebellar abnormalities may be due, at least in part, to the motoric demands of the task.

% DCN: I'm leaving this out because the whole mirror neuron system is now
% contentious, and I don't think we are citing specific evidence that shows
% mirror neuron damage in ASD.
%
% A theory which has recently caught the attention of the scientific community and the general population alike suggests that TOM and ``mind reading'' deficits in people with autism may result from problems in the ``mirror neuron'' system.  Mirror neurons, initially discovered in the F5 area of premotor cortex in non-human primates,  fire selectively when the animal performs specific goal directed actions~\cite{RefWorks:140}.  Interestingly, these same cells would also respond to the monkey watching another monkey or human performing the same action.  The discovery of cells which appear to respond symmetrically to a performed and observed action have led some researchers to suggest that the mirror neuron system could be vital for imitation skills and even TOM abilities in humans, and damaged mirror systems could therefore help explain the social deficits found in people with autism~\cite{RefWorks:141,RefWorks:138}.     

Structural MRI measurements of overall brain volume in people with ASD have found increased cerebral (white matter) volumes in people with ASD~\cite{FilipekPA:1995:AutismCerebellumMRI}, which are thought to arise from a failure in cortical pruning early in development~\cite{Eigsti:2003:AutismNeuroReview}. One study investigating brain growth in ASD reported that children with autism have larger brains than normal early in development (12 years and younger), but, by adolescence, these have returned to the size of age matched controls~\cite{RefWorks:142}. It is not immediately clear what effects the overgrowth of neural connections would have on behavior, but some theories suggest that these might include the rigid and context specific patterns of behavior seen in ASD~\cite{CohenIL:1994:AutismLearning}.

The limbic system is also of interest in autism research, largely due to its suggested role in social and emotional behavior. Postmortem anatomical studies have found abnormalities in the limbic systems of autistic brains~\cite{RefWorks:133}. Also, controlled damage to the amygdala has provided an interesting animal model of autism~\cite{BachevalierJ:1994:AutismAnimalModel}. Other areas of the limbic system, including the hippocampus and the hypothalamus, have been implicated as important for explaining behavioral differences in ASD~\cite{RefWorks:133}. 

Inspired partly by links to executive function deficits in ASD and partly by neuroanatomical findings, the PFC is an area of key interest for many researchers. Anatomically, researchers have identified ``narrow mini-columns'' in the PFC~\cite{Casanova:2003:AutismMiniColumns}, and they have noted that the parietal, temporal, and occipital lobes show overall brain volume enlargements, while the frontal lobes show no such increase. This means that the frontal lobes may be considered smaller in volume in comparison to the rest of the brain~\cite{PivenJ:1996:AutismBrainBig}. Considering the many executive function problems observed in ASD, the PFC stands out as a likely player in at least some of the unusual behaviors displayed in autism.

Moving beyond specific brain areas, diffuse neurotransmitter systems are also of interest in the search for the neural basis of ASD. Using techniques as diverse as urinalysis and PET imaging, differential amounts of serotonin and dopamine have been identified in people with autism~\cite{FernellE:1997:AutismPET,MartineauJ:1992:AutismDopamine,PoseyDJ:2000:AutismDopamine,RefWorks:72}. Clinical trials investigating the efficacy of drugs affecting levels of dopamine and serotonin in the brain have had mixed results~\cite{MartineauJ:1992:AutismDopamine,PoseyDJ:2000:AutismDopamine,Chugani:2004:AutismSerotonin}. Other differences in brain chemistry have been identified, including glutamatergic, GABA, and cholinergic abnormalities~\cite{RefWorks:95,RefWorks:96,RefWorks:137}. Also, oxytoxin and vasopression have been investigated as a possible source of problematic behavior in people with ASD~\cite{RefWorks:95,RefWorks:75}.

Given this extensive array of neurological differences in ASD, it is not surprising that strictly neuroscientific approaches, to date, have had little success in providing a unifying view of the biological mechanisms responsible for the patterns of behavior observed in autism. Even with extensive knowledge concerning differences in the developing brain of people with autism, it is difficult to understand exactly how these neuroscientific differences give rise to the complex behavioral profile of ASD.

Computational cognitive neuroscience modeling can be a useful tool for addressing this difficulty. What is needed is the fabrication and analysis of computational models of neural processes that can simulate the generation of behaviors comparable to those observed in the laboratory. The methods of computational cognitive neuroscience produce formal characterizations of the relationship between brain and behavior, entailing precise and testable hypotheses involving both neuroscientific and psychological measures. By offering explicit mechanistic accounts of the underlying neurobiology, while capturing actual behavioral patterns, computational cognitive neuroscience models provide a means of bridging the conceptual gap between cognitive psychology and cognitive neuroscience in ASD research.

One question previously explored using computational cognitive neuroscience techniques is how deliberate control over behavior (cognitive control) is instantiated within neural circuitry and how this control is adjusted as environmental contingencies change (cognitive flexibility). The prefrontal cortex (PFC) has been broadly implicated in both cognitive control and cognitive flexibility~\cite{Stuss:2000:WCSTLesion,Stuss:2001:StroopLesion}. Under some accounts, cognitive control involves the active maintenance of abstract rule-like representations in PFC~\cite{NoelleDC:2012:Rules}. These PFC representations provide a top-down task-appropriate processing bias to more posterior brain areas~\cite{CohenJD:1990:Stroop,RefWorks:154}. Biologically, the active maintenance of frontal control representations is supported by dense patterns of recurrent excitation in the PFC~\cite{PucakML:1996:Stripes}, as well as intrinsic maintenance currents~\cite{Goldman-RakicPS:1987:PFC_Maintenance}. Computational analyses have shown that cognitive control and cognitive flexibility are, in a sense, at odds. Cognitive control requires robust maintenance of a control representation, while cognitive flexibility requires the ability to quickly adapt these representations as task contingencies change. This processing conflict suggests the need for a mechanism to intelligently toggle the PFC between a maintenance mode and an updating mode. The fact that we learn to control our behavior in different ways depending on the current situation means that this toggling process must be learned. This need for learning has drawn the attention of researchers to the dopamine system.   

Dopamine (DA), a neurotransmitter with diffuse projections throughout the brain, plays a central role in contemporary models of PFC function. The mesolimbic DA system is seen as implementing a reinforcement learning algorithm, driving the learning of action sequences that lead to reward~\cite{MontaguePR:1996:Dopamine,BartoAG:1994:TDLearning}. In PFC models, the DA system learns to adjust the state of PFC pyramidal cells, determining when cognitive control should be maintained and when it should be flexibly modified in order to succeed at the current task~\cite{BraverTS:2000:Control,RougierNP:2005:XT}. A useful analogy is that of a ``gate'' in a fenced enclosure. When cognitive control must be strong, the gate is closed, keeping out distracting inputs that might compromise the current PFC control signals. When the current control state is no longer appropriate, the gate opens, allowing the old control state to escape and permitting a new control representation to enter the PFC via its inputs.  Recent computational models of PFC function suggest that intelligent ``gating'' of control representations in PFC can be learned, through experience, via the DA system~\cite{RougierNP:2005:XT,RougierNP:2002:TaskSwitching}. 

We propose that these computational accounts of PFC/DA interactions are highly relevant for understanding the neural basis of ASD. There is growing evidence for abnormal DA functioning in people with autism. Aberrant levels of DA have been discovered in studies measuring DA via PET~\cite{FernellE:1997:AutismPET}, as well as more indirect measures such as HVA metaboloites~\cite{MartineauJ:1992:AutismDopamine}. Clinical trials of drugs that modulate levels of DA in the brain have shown some behavioral benefits, as well~\cite{PoseyDJ:2000:AutismDopamine,TsaiLY:1999:AutismDopamine}. Studies have also found evidence of genetic differences in some kinds of dopamine receptors and morphological differences in the basal ganglia in people with ASD~\cite{deJongeM:2012:DopamineASDGenetics,MostofskySH:2010:BasalGangliaASDMotor}. DA system dysfunction is also associated with behaviors related to ASD symptoms, incuding increased prevalence of seizures~\cite{StarrMS:1996:SeizuresDA,TuchmanR:2002:EpilepsyAutism}, repetitive behaviors~\cite{CanalesJJ:2000:Stereotypy,RalphRJ:2001:Perseveration,Ralph-WilliansRJ:2003:HyperactiveDA}, and problems with skilled motor function~\cite{RinehartNJ:2001:AutismMovement,RinehartNJ:2006:AutismGait}.
% \cite{RefWorks:1,RefWorks:3,RefWorks:5,RefWorks:2,RefWorks:109}.
While the precise causal role that DA may play in autism is still unknown, the ties between DA and ASD are both numerous and compelling.

In this chapter, we report computational cognitive neuroscience simulation results supporting the conjecture that deficits in PFC/DA interactions are responsible for many of the interesting behavioral patterns observed in ASD. The basic idea is that reduced efficacy of the DA-based PFC gating mechanism results in overly perseverative top-down control. This has three results. First, the inability to properly adapt PFC control representations produces inflexible behavior. This is exemplified by the executive dysfunction profile observed in autism. The second consequence is more subtle. We suggest that the flexible updating of PFC plays an important role in shaping associational areas of cortex, influencing synaptic plasticity in these areas so as to support the appropriate generalization of learned behaviors across contexts. The hypothesized lack of flexible updating of PFC in ASD results in the learning of cortical representations that are overly specific, hindering generalization. These learning deficits can be seen in measures of prototype extraction during category learning. Third, failure to appropriately update PFC can limit the use of temporally extended context information, explaining observed deficits in sequential implicit learning tasks and in the use of sentential context to disambiguate the meaning of words.

% In the following we use computational modeling methods to investigate and formalize what effect a DA deficit would have on the behavior of a developing individual, relating simulated behavioral results to actual data from studies of people with autism.  

% In this chapter we investigate the degree to which a single deficit, a dysfunctional dopaminergic system, can account for many of the patterns of behavior demonstrated by people with autism.  This investigation makes use of the methods of computational cognitive neuroscience, producing formal models that demonstrate how dopamine dysfunction can produce the behavioral patterns of interest.  By focusing on a single neurological mechanism with diverse behavioral effects, this approach provides a level of inter-theoretic reduction not found in many current theories seeking to explain autism.  The implications of the research expand beyond fundamental theorizing, potentially providing an improved understanding of the successes and failures of interventions utilized to reduce overselectivity and foster the generalization of learned behaviors in people with ASD.

We begin by reviewing the processes of DA-mediated PFC updating. We then present a series of computational cognitive neuroscience models that demonstrate how dysfunctional PFC/DA interactions can account for experimental data concerning ASD behavioral patterns in the diverse domains of executive function, implicit learning, lexical dismbiguation, and prototype formation.
