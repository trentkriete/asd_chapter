%
% Conclusion
%

As awareness and resources continue to grow, so does the overall number of diagnoses of people with ASD.  At the same time, ASD continues to pose a massive challenge to researchers.  While great progress is being made in areas such as early identification of ASD, as well as intervention techniques, to date there is no sign of a converging consensus as to the true neural underpinnings of ASD.  Further complicating our understanding of the mechanisms that may underly autism is a staggeringly diverse behavioral profile as well as multiple physical abnormalities that often accompany a diagnosis.  In this document, we have presented a program of research in an attempt to address some of these issues.  Dopamine has diffuse and widespread effects of throughout the human cortex. This coupled with strong ties to multiple clinical populations as well as numerous ties to human behavior make it an intriguing initial candidate for a disorder possessing a profile like that of autism.  However, it is not until we recognize the myriad of ties between DA and core autistic behavioral differences that we begin to see the true potential of the DA dysfunction hypothesis of ASD.  Increased seizure rates, motor abnormalities, stereotyped and repetitive behaviors, executive dysfunction, abnormal gaits, problems learning to follow eye gaze, and attentional abnormalities are all key components of behavior in autism, and all are linked tightly to the mid-brain dopamine system~\cite{RefWorks:99,RefWorks:100,RefWorks:102,HillEL:2004:AutismExecutiveDysfunction,Ozonoff:1991:AutismExecutiveDysfunction,RefWorks:1,RefWorks:3,RefWorks:5,RefWorks:2,RefWorks:109}, supporting the argument for a role for the dopaminergic system in the etiology of autism. Importantly, we have argued that perturbed DA / PFC interactions may lead to overly perseverative attention in autism, providing a neurally precise and plausible mechanism that might link previously disjoint theories in autism research. 

By separating the mechanisms responsible for cognitive control and the flexible adjustment of control, perplexing aspects of the specific executive dysfunction profile demonstrated by people with autism are nicely captured.  Cognitive control is instantiated via actively maintained control representations within the prefrontal cortex.  Cognitive flexibility is implemented via interactions between PFC and the mid-brain dopamine system.  These interactions are suspect in autism, resulting in problems in flexibly updating the control instantiated via the PFC and capturing the problematic profile demonstrated by people with ASD~\cite{HillEL:2004:AutismExecutiveDysfunction,Ozonoff:1991:AutismExecutiveDysfunction}.  Developmentally, executive dysfunction does not appear until later in childhood.  My modeling efforts indicate that this may occur due to the protracted development of the PFC.  In my model, early performance is driven largely by non-frontal, more posterior, brain systems which are largely unaffected by the posited DA-related abnormalities in autism.  As the PFC becomes more effective, differences in PFC/DA interactions are unmasked.  

Stimulus overselectivity, where a restricted subset of possible items or features in the environment dominate behavior in people with ASD, can also be subsumed under the same theoretical framework.  We hypothesize that frequent and flexible updating of the attentional and control representations stored in the PFC is necessary in order to prevent an overly restricted subset of items in the environment from gaining control over our behavior.  Under this account, inflexible and infrequent updating results in a restricted subset of features from the environment dominating the contents maintained within the PFC.  Subsequently, through an associative learning process, the restricted subset comes to possess stronger ``association weights'' and thus dominate responding compared to the other features in the environment.

In another area of interest, weak central coherence theorists posit that people with autism have difficulties integrating pieces of information into a coherent whole or ``gestalt''\cite{FrithU:1989:AutismWCC,HappeF:1999:WCC}.  A major contribution of this work is demonstrating how top-down PFC-like mechanisms may influence the representations learned in other cortical areas.  As such, WCC may be recast not as a problem in the integration of information per se, but rather as integrating the \emph{wrong} information, due to the inflexible updating of attentional / control representations stored within the PFC.   Problems with implicit learning as well as using contextual information to disambiguate sentential context can both be explained utilizing this account.  Tasks investigating implicit learning, such as the Serial Response Time Task (SRTT), depend on previous information about the sequence to be readily available on subsequent time steps in order for normal learning to occur.  Without reliable contextual information, neural systems will struggle to integrate the past information in an appropriate manner.  The same is true when determining the meaning of ambiguous words in a sentence, without an appropriate representation of the context, the best we can do is to rely on the statistical frequency of the words we experience.   

Finally, people with autism also demonstrate an atypical prototype effect when learning category structures~\cite{RefWorks:113,StraussMS:2009:Prototype}.   It is reasonable to assume that in order to correctly form and use a prototype, we must have the ability to spread our attention out somewhat evenly across the relevant features of a stimulus or category example.  If, as would be caused by inflexible attentional of the PFC, we highlight and learn to ``over-value'' a restricted subset of the features, the representation that is learned would likely not represent the standard mathematical average of psychological feature values as is argued to occur during prototype formation.  In this case, an individual may become ``overselective'' and weight the restricted subset of feature values more highly during category determination.  



The convergence of data supporting dopamine's role in autism, combined with the possibility of providing a conceptual bridge spanning multiple theories in autism, is extremely encouraging.  Computational modeling results presented in this document help to demonstrate the potential for formal computational models investigating the links between possible neural underpinnings and behavior in people with ASD.  Using simulations, constrained and informed by both biology and observed behavior, precise and testable predictions of underlying mechanisms can be made, providing a theoretical bridge between psychological and anatomic theories of ASD.   Namely, my modeling efforts suggest that dysfunctional interactions between the mid-brain DA system and the PFC may lead to overly perseverative top-down attentional effects in people with ASD.  By casting the PFC as key player in both attention and in the shaping of posterior cortical areas, my theory provides a way to unify multiple previously disparate behavioral phenomena observed in people with ASD.
