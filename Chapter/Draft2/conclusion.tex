%
% Conclusion
%

Autism diagnoses are on the rise. At the same time, ASD continues to pose a serious challenge to researchers. To date, there is no consensus concerning the neural underpinnings of ASD. Further complicating our understanding of autism is the staggeringly diverse behavioral profile of the disorder, as well as multiple physical abnormalities that often accompany a diagnosis. In this chapter, we have presented an approach intended to address some of these issues. Dopamine has diffuse and widespread effects throughout the brain. It has strong ties to multiple clinical populations. Increased seizure rates, motor abnormalities, stereotyped and repetitive behaviors, executive dysfunction, abnormal gaits, problems learning to follow eye gaze, and attentional abnormalities are all key components of behavior in autism, and all are linked tightly to the midbrain dopamine system. These things make dopamine dysfunction an intriguing candidate mechanism to consider. This chapter provides additional reasons to suspect a role for dopamine in autism. We have argued that perturbed DA/PFC interactions may lead to overly perseverative top-down control in autism, providing a neurally precise and plausible mechanism that might link previously disjoint theories in autism research. 

By separating the mechanisms responsible for cognitive control and the flexible adjustment of control, perplexing aspects of the executive dysfunction profile in ASD are nicely captured. Cognitive control is instantiated via actively maintained control representations in PFC. Cognitive flexibility is implemented via interactions between PFC and the DA system. These interactions are suspect in autism, resulting in problems specifically in the flexible and appropriate updating of control and capturing the problematic profile of executive dysfunction seen in ASD~\cite{HillEL:2004:AutismExecutiveDysfunction}. Developmentally, executive dysfunction appears late in childhood. Our modeling efforts indicate that this may occur due to the protracted development of the PFC. In our model, early performance is driven largely by non-frontal, more posterior, brain systems which are largely unaffected by the posited DA abnormalities in ASD. As the PFC becomes more effective, differences in PFC/DA interactions are unmasked.  

%Stimulus overselectivity, where a restricted subset of possible items or features in the environment dominate behavior in people with ASD, can also be subsumed under the same theoretical framework.  We hypothesize that frequent and flexible updating of the attentional and control representations stored in the PFC is necessary in order to prevent an overly restricted subset of items in the environment from gaining control over our behavior.  Under this account, inflexible and infrequent updating results in a restricted subset of features from the environment dominating the contents maintained within the PFC.  Subsequently, through an associative learning process, the restricted subset comes to possess stronger ``association weights'' and thus dominate responding compared to the other features in the environment.

This work also speaks to Weak Central Coherence theory, which posits that people with autism have difficulties integrating pieces of information into a coherent ``gestalt''~\cite{FrithU:1989:AutismWCC,HappeF:1999:WCC}. The simulations reported in this chapter demonstrate how top-down PFC modulation of neural processing can influence the representations learned in other cortical areas. As such, WCC may be recast from being a general problem of information integration to being a problem of integrating the \emph{wrong} information, due to the inflexible updating of PFC representations. Problems with implicit learning in the SRTT, as well as using sentential context to disambiguate word meanings, can both be explained by this account. These tasks depend on previously experienced information to be readily available at a later time in order for normal learning to occur, including learning that occurs at developmental time scales. Without reliable contextual information, neural systems struggle to integrate past information in an appropriate manner, driving learning to depend on other, less reliable, cues (e.g., frequency of word meaning).

We have also discussed differences in how people with autism learn category structures~\cite{RefWorks:113,StraussMS:2009:Prototype}. It is reasonable to assume that, in order to correctly form and use a prototype, we must have the ability to spread our attention across the relevant features of category examples. Learning to ``over-value'' a restricted subset of features results in a failure to discern a valid prototype, and inflexible updating of top-down attentional control from PFC can produce such a restricted focus. In this way, our account addresses at least some phenomena of \emph{stimulus overselectivity} observed in ASD~\cite{RefWorks:110,RefWorks:112}.

The convergence of evidence supporting dopamine's role in autism, combined with the possibility of providing a conceptual bridge spanning multiple theories in autism, is extremely encouraging. The work presented in this chapter helps to demonstrate the potential of computational cognitive neuroscience methods when investigating the links between biology and behavior in people with ASD.
